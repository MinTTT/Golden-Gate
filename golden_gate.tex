%%%%%%%%%%%%%%%%%%%%%%%%%%%%%%%%%%%%%%%%%
% University/School Laboratory Report
% LaTeX Template
% Version 3.1 (25/3/14)
%
% This template has been downloaded from:
% http://www.LaTeXTemplates.com
%
% Original author:
% Linux and Unix Users Group at Virginia Tech Wiki 
% (https://vtluug.org/wiki/Example_LaTeX_chem_lab_report)
%
% License:
% CC BY-NC-SA 3.0 (http://creativecommons.org/licenses/by-nc-sa/3.0/)
%
%%%%%%%%%%%%%%%%%%%%%%%%%%%%%%%%%%%%%%%%%

%----------------------------------------------------------------------------------------
%	PACKAGES AND DOCUMENT CONFIGURATIONS
%----------------------------------------------------------------------------------------

\documentclass[a4paper, twocolumn]{article}

\usepackage[version=3]{mhchem} % Package for chemical equation typesetting
\usepackage{siunitx} % Provides the \SI{}{} and \si{} command for typesetting SI units
\usepackage{graphicx} % Required for the inclusion of images
\usepackage{natbib} % Required to change bibliography style to APA
\usepackage{amsmath} % Required for some math elements 
\usepackage[a4paper, top=1cm, bottom=2.5cm, left=2.5cm, right=2.5cm, inner=1.5cm, bindingoffset=1cm]{geometry}
\usepackage{booktabs} % for table which have three line type
\setlength\parindent{0pt} % Removes all indentation from paragraphs
\usepackage{float}
\renewcommand{\labelenumi}{\alph{enumi}.} % Make numbering in the enumerate environment by letter rather than number (e.g. section 6)

%\usepackage{times} % Uncomment to use the Times New Roman font
\def\celsius{\ensuremath{^\circ\hspace{-0.09em}\mathrm{C}}}
\newcommand{\tabincell}[2]{\begin{tabular}{@{}#1@{}}#2\end{tabular}} 
%----------------------------------------------------------------------------------------
%	DOCUMENT INFORMATION
%----------------------------------------------------------------------------------------

\title{Golden Gate Assembly} % Title

\author{Pan M. \textsc{chu}} % Author name

\date{\today @ SIAT} % Date for the report

\begin{document}

\maketitle % Insert the title, author and date

\begin{center}
\begin{tabular}{l r}
Date Performed: & July 19, 2019 \\ % Date the experiment was performed
% Partners: & James Smith \\ % Partner names
% & Mary Smith \\
% Instructor: & Professor Fu \\ % Instructor/supervisor
\end{tabular}
\end{center}

% If you wish to include an abstract, uncomment the lines below
% \begin{abstract}
% Abstract text
% \end{abstract}

%----------------------------------------------------------------------------------------
%	SECTION 1
%----------------------------------------------------------------------------------------

% \section{Objective}

% To determine the atomic weight of magnesium via its reaction with oxygen and to study the stoichiometry of the reaction (as defined in \ref{definitions}):

% \begin{center}\ce{2 Mg + O2 -> 2 MgO}\end{center}

% % If you have more than one objective, uncomment the below:
% %\begin{description}
% %\item[First Objective] \hfill \\
% %Objective 1 text
% %\item[Second Objective] \hfill \\
% %Objective 2 text
% %\end{description}

% \subsection{Definitions}
% \label{definitions}
% \begin{description}
% \item[Stoichiometry]
% The relationship between the relative quantities of substances taking part in a reaction or forming a compound, typically a ratio of whole integers.
% \item[Atomic mass]
% The mass of an atom of a chemical element expressed in atomic mass units. It is approximately equivalent to the number of protons and neutrons in the atom (the mass number) or to the average number allowing for the relative abundances of different isotopes. 
% \end{description} 
 
% %----------------------------------------------------------------------------------------
% %	SECTION 2
% %----------------------------------------------------------------------------------------

% \section{Experimental Data}

% \begin{tabular}{ll}
% Mass of empty crucible & \SI{7.28}{\gram}\\
% Mass of crucible and magnesium before heating & \SI{8.59}{\gram}\\
% Mass of crucible and magnesium oxide after heating & \SI{9.46}{\gram}\\
% Balance used & \#4\\
% Magnesium from sample bottle & \#1
% \end{tabular}

% %----------------------------------------------------------------------------------------
% %	SECTION 3
% %----------------------------------------------------------------------------------------

% \section{Sample Calculation}

% \begin{tabular}{ll}
% Mass of magnesium metal & = \SI{8.59}{\gram} - \SI{7.28}{\gram}\\
% & = \SI{1.31}{\gram}\\
% Mass of magnesium oxide & = \SI{9.46}{\gram} - \SI{7.28}{\gram}\\
% & = \SI{2.18}{\gram}\\
% Mass of oxygen & = \SI{2.18}{\gram} - \SI{1.31}{\gram}\\
% & = \SI{0.87}{\gram}
% \end{tabular}

% Because of this reaction, the required ratio is the atomic weight of magnesium: \SI{16.00}{\gram} of oxygen as experimental mass of Mg: experimental mass of oxygen or $\frac{x}{1.31}=\frac{16}{0.87}$ from which, $M_{\ce{Mg}} = 16.00 \times \frac{1.31}{0.87} = 24.1 = \SI{24}{\gram\per\mole}$ (to two significant figures).

% %----------------------------------------------------------------------------------------
% %	SECTION 4
% %----------------------------------------------------------------------------------------

% \section{Results and Conclusions}

% The atomic weight of magnesium is concluded to be \SI{24}{\gram\per\mol}, as determined by the stoichiometry of its chemical combination with oxygen. This result is in agreement with the accepted value.

% \begin{figure}[h]
% \begin{center}
% \includegraphics[width=0.45\textwidth]{placeholder} % Include the image placeholder.png
% \caption{Figure caption.}
% \end{center}
% \end{figure}

% %----------------------------------------------------------------------------------------
% %	SECTION 5
% %----------------------------------------------------------------------------------------

% \section{Discussion of Experimental Uncertainty}

% The accepted value (periodic table) is \SI{24.3}{\gram\per\mole} \cite{Smith:2012qr}. The percentage discrepancy between the accepted value and the result obtained here is 1.3\%. Because only a single measurement was made, it is not possible to calculate an estimated standard deviation.

% The most obvious source of experimental uncertainty is the limited precision of the balance. Other potential sources of experimental uncertainty are: the reaction might not be complete; if not enough time was allowed for total oxidation, less than complete oxidation of the magnesium might have, in part, reacted with nitrogen in the air (incorrect reaction); the magnesium oxide might have absorbed water from the air, and thus weigh ``too much." Because the result obtained is close to the accepted value it is possible that some of these experimental uncertainties have fortuitously cancelled one another.

% %----------------------------------------------------------------------------------------
% %	SECTION 6
% %----------------------------------------------------------------------------------------

% \section{Answers to Definitions}

% \begin{enumerate}
% \begin{item}
% The \emph{atomic weight of an element} is the relative weight of one of its atoms compared to C-12 with a weight of 12.0000000$\ldots$, hydrogen with a weight of 1.008, to oxygen with a weight of 16.00. Atomic weight is also the average weight of all the atoms of that element as they occur in nature.
% \end{item}
% \begin{item}
% The \emph{units of atomic weight} are two-fold, with an identical numerical value. They are g/mole of atoms (or just g/mol) or amu/atom.
% \end{item}
% \begin{item}
% \emph{Percentage discrepancy} between an accepted (literature) value and an experimental value is
% \begin{equation*}
% \frac{\mathrm{experimental\;result} - \mathrm{accepted\;result}}{\mathrm{accepted\;result}}
% \end{equation*}
% \end{item}
% \end{enumerate}

%----------------------------------------------------------------------------------------
%	BIBLIOGRAPHY
%----------------------------------------------------------------------------------------
\section{Introduction}
Pending...
\section{Protocol}
\subsection{Reaction System}
As an example, here, we use BsaI (type IIs restriction enzyme) in this reaction system. Below is an typical reaction system which is $20 \ \mathrm{\mu L}$.
\begin{table}[H]
    \centering
    \begin{tabular}{lp{3cm}}
        \toprule[1.5pt]
        Components & Volume \\
        \midrule 
        Fragments  & $40$ fmol or $100$ ng ea. $^{(1)}$\\
        BsaI & $10 \ \mathrm{U}$ ($2 \ \mathrm{\mu L}$)\\
        T4 ligase & $5 \ \mathrm{U}$ ($0.5\ \mathrm{\mu L}$) $^{(2)}$\\
        $10 \times$ T4 Ligase Buffer &  $2 \ \mathrm{\mu L}$ \\
        $\mathrm{ddH_2O}$ & Up to $20 \ \mathrm{\mu L}$  \\
        \bottomrule[1.5pt]
    \end{tabular}
    \caption{Golden Gate reaction system}
\end{table}

\textbf{Note:}
\begin{enumerate}
    \item[(1)] When do $1 \sim 2$ insertions reaction, it is convenient adding $100 $ ng of each fragment into reaction. If performing multiple ($> 3$ insertions) ligation, we recommend adding $40 $ fmol of each fragment.
    \item[(2)] If assembling more than 3 fragments, high concentration of T4 ligase is necessary. I recommend 20 U (for NEB CEU, 1 U = 200 CEU). 
    \item[(3)] For convenience, we often prepare 10 $\mathrm{\mu L}$ reactions and the volume of each enzyme is 1 $\mathrm{\mu L}$.
\end{enumerate}

\subsection{Ligation Procedure}
If performing one fragment insertion ligation, incubating reaction system $30 \ \textrm{min}$ at $37 \ \celsius$. When insertions are more than 2, we recommend using thermocycler: $37\ \celsius$ for $2$ min, the $16 \ \celsius$ for 3 min, repeat for 10 - 100 cycles, depending on the number of insertions (10 cycles per insertion). Following, the reaction is heated to $50 \ \celsius$ for 30 min, 30 min at 37 \celsius \ , which is for digesting non-specific ligation events, completely. Last, the reaction is incubated 5 min at $80 \ \celsius$ for inactivating the reaction enzyme. (Optionally, 50 $\mathrm{\mu L}$ system containing 1 $\mathrm{\mu L}$ T4 DNA ligase (2, 000, 000 $\mathrm{U / \mu L}$), 2 $\mathrm{\mu L}$ Type IIs restriction enzyme and totaling $\sim 1\mathrm{\mu g} $ DNA, is recommended. After reaction, 1 $\mathrm{\mu L}$ additional restriction enzyme is added, and the assembly is incubated for 1 h at 27 \celsius. Assembly is then purified for electroporation.) The 10 $\mathrm{\mu L}$ reaction is transformed to 50 $\mathrm{\mu L}$ chemically competent cell. Then stalled cell in sawed ice for 30 min, and allowed to recover for an hour in SOC medium. 

\section{Design}

\subsection{Primer Design}
The flank sequences of primer for introducing enzymatic cutting site into double ends of insertions are collected in Table \ref{flank seq}.
\begin{table}[H]
    \centering
    \begin{tabular}{lp{4.7cm}}
        \toprule[1.5pt]
        Enzyme Name & Sequence \\
        \midrule
        BsaI & GCATTA\underline{GGTCTC}CNNNN \\
        BbsI & AAGTGC\underline{GAAGAC}CANNNN \\
        BfuAI & \tabincell{l}{GGCAAT\underline{ACCTGC}GTGANN\\NN} \\
        BtgZI & \tabincell{l}{CGAATG\underline{GCGATG}TTG\\TACTGCCNNNN} \\
        SapI & GGAATC\underline{GCTCTTC}CNNN \\
        BsmBI & ATAGCG\underline{CGTCTC}CNNNN \\
        \bottomrule[1.5pt]
    \end{tabular}
    \caption{Flank Sequence of Primer, the underlined nucleotides are denoted as the enzyme recognition sequence, and the sequence before recognition sequence is protection sequence, and the base N is denote as the sticky end.}\label{flank seq}
\end{table}
\bibliographystyle{apalike} 

% \bibliography{sample}

%----------------------------------------------------------------------------------------


\end{document}